% Created 2021-09-19 Sun 15:19
% Intended LaTeX compiler: pdflatex
\documentclass{article}
\usepackage[utf8]{inputenc}
\usepackage[T1]{fontenc}
\usepackage{graphicx}
\usepackage{grffile}
\usepackage{longtable}
\usepackage{wrapfig}
\usepackage{rotating}
\usepackage[normalem]{ulem}
\usepackage{amsmath}
\usepackage{textcomp}
\usepackage{amssymb}
\usepackage{capt-of}
\usepackage{hyperref}

\usepackage[a4paper,left=0.5in,right=0.5in,top=0.5in,bottom=1in]{geometry}
\usepackage{float}
\DeclareUnicodeCharacter{2212}{-}
\setcounter{secnumdepth}{0}
\author{Tzong Lin Chua}
\date{\today}
\title{EE4C10 Analog Circuit Design Fundamentals\\\medskip
\large Homework Assignment II }
\hypersetup{
 pdfauthor={Tzong Lin Chua},
 pdftitle={EE4C10 Analog Circuit Design Fundamentals},
 pdfkeywords={},
 pdfsubject={},
 pdfcreator={Emacs 27.1 (Org mode 9.5)}, 
 pdflang={English}}
\begin{document}

\maketitle
\tableofcontents


\section{Problem 1}
\label{sec:org30ad0b8}
\begin{enumerate}
\item Overdrive voltage, V\textsubscript{gt}, for:
\begin{enumerate}
\item M1
\begin{equation*}
\begin{aligned}
I_{D1} &= \frac{\mu_{n}C_{OX}}{2}(\frac{W}{L})_{1}(V_{GS_{1}} - V_{TH_{1}})^2(1 + \lambda_{1}V_{DS_{1}}) \\
I_{D1} &\approx \frac{\mu_{n}C_{OX}}{2}(\frac{W}{L})_{1}(V_{gt_{1}})^2 \\
V_{gt_{1}} &\approx \sqrt{\frac{2 I_{D_{1}}}{\mu_{n}C_{OX}}(\frac{L}{W})_{1}} \\
\\
V_{gt_{1}} &\approx 109.11 mV
\end{aligned}
\end{equation*}

\item M2
\begin{equation*}
\begin{aligned}
V_{gt_{2}} &\approx \sqrt{\frac{2 I_{D_{2}}}{\mu_{p}C_{OX}}(\frac{L}{W})_{2}} \\
\\
V_{gt_{2}} &\approx 377.96 mV
\end{aligned}
\end{equation*}
\end{enumerate}

\item Small-signal gain

\begin{equation*}
\begin{aligned}
g_{m1}V_{in} &= \frac{-V_{out}}{r_{o1}//r_{o2}} \\
\frac{V_{out}}{V_{in}} &= -g_{m1}(r_{o1}//r_{o2}) \\
\\
g_{m1} &= \mu_{n}C_{OX} (\frac{W}{L})_{1} V_{gt_1} \\
&= 4.582 mS \\
\\
r_{o1} &= \frac{1}{I_{D1}\lambda_{n}} \\
&= 20 k\Omega \\
\\
r_{o2} &= \frac{1}{I_{D2}\lambda_{p}} \\
&= 40 k\Omega \\
\\
\frac{V_{out}}{V_{in}} &\approx -61.09 \\
\end{aligned}
\end{equation*}

\item V\textsubscript{out} output swing

For M\textsubscript{1} to be in saturation,
\begin{equation*}
\begin{aligned}
V_{DS1} &\geq V_{gt1}\\
V_{out} &\geq 0.109 V
\end{aligned}
\end{equation*}

For M\textsubscript{2} to be in saturation,
\begin{equation*}
\begin{aligned}
V_{DS2} &\geq V_{gt2} \\
V_{DD} - V_{out} &\geq 0.377 V \\
V_{out} &\leq 3.3 V - 0.377 V \\
V_{out} &\leq 2.923 V \\
\end{aligned}
\end{equation*}

Swing of V\textsubscript{out},
\begin{equation*}
\begin{aligned}
0.109 V &< V_{out} < 2.923 V \\
\\
V_{out, pp} &= 2.923 V - 0.109 V \\
&= 2.814 V
\end{aligned}
\end{equation*}

\item 
\end{enumerate}

\section{Problem 2}
\label{sec:orgd74c234}
\begin{enumerate}
\item For M1 to be 100mV from triode,
\begin{equation*}
\begin{aligned}
V_{DS1} &= V_{GS1} - V_{TH,N} + 100mV \\
X &= V_{in} - V_{TH,N} + 100mV \\
\end{aligned}
\end{equation*}
V\textsubscript{in} for M1 to be in saturation with I\textsubscript{D1} of 0.35 mA,
\begin{equation*}
\begin{aligned}
I_{D1} &= \frac{\mu_{n}C_{OX}}{2}(\frac{W}{L})_{1}(V_{GS1} - V_{TH,N})^2 \\
I_{D1} &= \frac{\mu_{n}C_{OX}}{2}(\frac{W}{L})_{1}(V_{in} - V_{TH,N})^2 \\
V_{in} &= \sqrt{\frac{2I_{D1}}{\mu_{n}C_{OX}}(\frac{L}{W})_{1}} + V_{TH,N} \\
\\
X &= \sqrt{\frac{2I_{D1}}{\mu_{n}C_{OX}}(\frac{L}{W})_{1}} + 100mV \\
&\approx 0.253 V
\end{aligned}
\end{equation*}
V\textsubscript{b} for M2 to be in saturation with I\textsubscript{D2} of 0.35 mA,
\begin{equation*}
\begin{aligned}
I_{D2} &= \frac{\mu_{n}C_{OX}}{2}(\frac{W}{L})_{2}(V_{GS2} - V_{TH,N})^2 \\
I_{D2} &= \frac{\mu_{n}C_{OX}}{2}(\frac{W}{L})_{2}(V_{b} - X - V_{TH,N})^2 \\
V_{b} &= \sqrt{\frac{2I_{D2}}{\mu_{n}C_{OX}}(\frac{L}{W})_{2}} + X + V_{TH,N} \\
&\approx 0.906 V

\end{aligned}
\end{equation*}

\item Small-signal gain

\begin{equation*}
\begin{aligned}
G_{m} = \frac{-g_{m1}}{1 + \frac{r_{o2}}{r_{o1}}} \\
\end{aligned}
\end{equation*}
\begin{equation*}
\begin{aligned}
R_{out} &= (r_{o1} + r_{o2}) // R_{d} \\
\end{aligned}
\end{equation*}

Small-signal gain,
\begin{equation*}
\begin{aligned}
\frac{V_{out}}{V_{in}} &= G_{m}R_{out} \\
&= \frac{-g_{m1}r_{o1}R_{d}}{r_{o1}+r_{o2}+R_{d}} \\
\\
g_{m1} &= \mu_{n}C_{OX} (\frac{W}{L})_{1} (V_{GS1} - V_{TH,N})(1 + \lambda_{n}V_{DS1}) \\
&= \mu_{n}C_{OX} (\frac{W}{L})_{1} (V_{in} - V_{TH,N})(1 + \lambda_{n}}X) \\
&= 4.698 mS \\
\\
r_{o1} &= \frac{1}{I_{D1}\lambda_{n}} \\
&= 28.571 k\Omega \\
\\
r_{o2} &= \frac{1}{I_{D2}\lambda_{p}} \\
&= 28.571 k\Omega \\
\\
\frac{V_{out}}{V_{in}} &\approx -10.90 \\
\\
\end{aligned}
\end{equation*}

\item Maximum output swing,
\end{enumerate}
\end{document}
